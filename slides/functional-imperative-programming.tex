{
  \newmdenv[tikzsetting={draw=black,fill=white,fill opacity=0.7, line width=4pt},backgroundcolor=none,leftmargin=0,rightmargin=0,innertopmargin=4pt,skipbelow=\baselineskip,%
  skipabove=\baselineskip]{TitleBoxFunctionalImperativeProgramming}

  \usebackgroundtemplate{\includegraphics[width=1.0\paperwidth]{image/title-background.png}}

  \begin{frame}[plain] 
  \title{Functional Imperative Programming}
  
  \vspace{3em}

  \begin{TitleBoxFunctionalImperativeProgramming}
    \begin{center}
    {\Large \inserttitle}
    \end{center}
  \end{TitleBoxFunctionalImperativeProgramming}

  \end{frame}
}


\begin{frame}
\frametitle{Functional v Imperative v OOP}
\framesubtitle{The fallacy of false compromise}
\begin{block}{A note on OOP}
I am going to dismiss Object-Oriented Programming, because I don't know what it is \tiny{and neither do you}
\end{block}
\end{frame}


{
\usebackgroundtemplate{\includegraphics[width=1.0\paperwidth]{image/book.png}}
\begin{frame}
\frametitle{Functional v Imperative}
\begin{center}
\LARGE{The parable of he who is not even wrong}
\end{center}
\end{frame}
}

{
\usebackgroundtemplate{
\begin{tikzpicture}[remember picture,overlay]
  \coordinate (aa) at ($(a1)+(-1,7.5)$);
  \node[right] at (aa) {\includegraphics[height=1cm]{image/book-small.png}};
\end{tikzpicture}
}

\begin{frame}
\frametitle{Functional v Imperative}
\framesubtitle{The parable of he who is not even wrong}
Someone in a pub once said to me, not too long ago
\begin{quote}
Hi my name is Wiggleydoo and I am the Chief Python Wippedy-wop for Hoopdiddy-zip
\end{quote}
\end{frame}


\begin{frame}
\frametitle{Functional v Imperative}
\framesubtitle{The parable of he who is not even wrong}
and I thought to myself, ``oh yeah that's nice''
\end{frame}


\begin{frame}
\frametitle{Functional v Imperative}
\framesubtitle{The parable of he who is not even wrong}
and then this happened
\begin{quote}
Functional programming is great and all, but I only use state where it is appropriate. You know\ldots when the problem demands stateful things.
\end{quote}
\end{frame}


\begin{frame}
\frametitle{Functional v Imperative}
\framesubtitle{The parable of he who is not even wrong}
\begin{center}
and it all came back to me
\end{center}
\begin{center}
\includegraphics[width=0.7\paperwidth]{image/shock.jpg}
\end{center}
\end{frame}


\begin{frame}
\frametitle{Functional v Imperative}
\framesubtitle{The parable of he who is not even wrong}
\begin{block}{Fact}
There is no such thing as an ``inherently stateful'' computation or algorithm
\end{block}
\end{frame}


\begin{frame}
\frametitle{Functional v Imperative}
\framesubtitle{The parable of he who is not even wrong}
\begin{block}{So I called shenanigans on that}
\begin{quote}
but what about those algorithms that demand imperative programming?
\end{quote}
\end{block}
\end{frame}


\begin{frame}
\frametitle{Functional v Imperative}
\framesubtitle{The parable of he who is not even wrong}
\begin{block}{Church-Turing Thesis}
\begin{quote}
Many people do imperative programming using pure-functional programming all day, every day
\end{quote}
This can be achieved for every program that can possibly exist
\end{block}
\end{frame}


\begin{frame}
\frametitle{Functional v Imperative}
\framesubtitle{The parable of he who is not even wrong}
\begin{block}{``but how?''}
At this point I am stumped. ``Casually \ldots er neatly?''
\end{block}
\end{frame}


\begin{frame}
\frametitle{Functional v Imperative}
\framesubtitle{The parable of he who is not even wrong}
\begin{block}{Recognising my confoundment}
My friend pulled out some imperative pure-functional production code that had been written at a bank
\end{block}
\end{frame}


\begin{frame}
\frametitle{Functional v Imperative}
\framesubtitle{The parable of he who is not even wrong}
\begin{block}{Escaping the state of delirium}
The discussion then quickly turned to beer
\end{block}
\end{frame}
}


\begin{frame}[fragile]
\frametitle{Functional v Imperative}
\begin{block}{Here is an imperative Haskell program}
\begin{lstlisting}[style=haskell,mathescape]
program = do
  a <- readFile "file"
  print a
  writeFile "cods!" "file"
  b <- readFile "file"
  print b
\end{lstlisting}
\end{block}
\end{frame}


\begin{frame}[fragile]
\frametitle{Functional v Imperative}
\begin{block}{Here is an imperative Haskell program}
\begin{lstlisting}[style=haskell,mathescape]
program = do
  a <- `readFile "file"`
  print a
  writeFile "cods!" "file"
  b <- `readFile "file"`
  print b
\end{lstlisting}
\end{block}
\begin{tikzpicture}[remember picture,overlay]
\coordinate (aa) at ($(a1)+(5,5.8)$);
\node[note,draw,callout relative pointer={($(aa)-(5.7,4.0)$)},right] at (aa) {well?};
\node[note,draw,callout relative pointer={($(aa)-(5.9,5.1)$)},right] at (aa) {well?};
\end{tikzpicture}
\end{frame}


\begin{frame}[fragile]
\frametitle{Functional v Imperative}
\begin{block}{Here is an imperative Haskell program}
\begin{lstlisting}[style=haskell,mathescape]
file =
  readFile "file"

program = do
  a <- `file`
  print a
  writeFile "cods!" "file"
  b <- `file`
  print b
\end{lstlisting}
\end{block}
\begin{tikzpicture}[remember picture,overlay]
\coordinate (aa) at ($(a1)+(5,4.2)$);
\node[note,draw,callout relative pointer={($(aa)-(7.8,2.4)$)},right] at (aa) {did the program change?};
\node[note,draw,callout relative pointer={($(aa)-(8.3,3.2)$)},right] at (aa) {did the program change?};
\end{tikzpicture}
\end{frame}


\begin{frame}[fragile]
\frametitle{Functional v Imperative}
\begin{block}{Here is an imperative Haskell program}
\begin{lstlisting}[style=haskell,mathescape]
file =
  readFile "file"

program = do
  `a <- file`
  `print a`
  writeFile "cods!" "file"
  `b <- file`
  `print b`
\end{lstlisting}
\end{block}
\begin{tikzpicture}[remember picture,overlay]
\coordinate (aa) at ($(a1)+(5,4.2)$);
\node[note,draw,callout relative pointer={($(aa)-(8.0,2.6)$)},right] at (aa) {in fact, look at this repetition of work};
\node[note,draw,callout relative pointer={($(aa)-(9.8,3.7)$)},right] at (aa) {in fact, look at this repetition of work};
\end{tikzpicture}
\end{frame}


\begin{frame}[fragile]
\frametitle{Functional v Imperative}
\begin{block}{Here is an imperative Haskell program}
\begin{lstlisting}[style=haskell,mathescape]
printfile = do
  f <- readFile "file"
  print f

program = do
  `printfile`
  writeFile "cods!" "file"
  `printfile`
\end{lstlisting}
\end{block}
\end{frame}


\begin{frame}[fragile]
\frametitle{Functional v Imperative}
\begin{block}{Did I mention that functional programming is}
\begin{center}
\textbf{Don't Repeat Yourself} without the duplicity?
\end{center}
\end{block}
\end{frame}


\begin{frame}[fragile]
\frametitle{Functional v Imperative}
\begin{block}{Here is an imperative Haskell program}
\begin{lstlisting}[style=haskell,mathescape]
printfile = do
  f <- readFile "file"
  print f

program = do
  `printfile`
  writeFile "cods!" "file"
  `printfile`
\end{lstlisting}
\end{block}
\begin{tikzpicture}[remember picture,overlay]
\coordinate (aa) at ($(a1)+(7,4.2)$);
\node[note,draw,callout relative pointer={($(aa)-(11.8,2.6)$)},right] at (aa) {stop it!};
\node[note,draw,callout relative pointer={($(aa)-(11.8,3.3)$)},right] at (aa) {stop it!};
\end{tikzpicture}
\end{frame}


\begin{frame}[fragile]
\frametitle{Functional v Imperative}
\begin{block}{Here is an imperative Haskell program}
\begin{lstlisting}[style=haskell,mathescape]
a >.> b = do
  a
  b
  a
  
printfile = do
  f <- readFile "file"
  print f

program =
  printfile >.> writeFile "cods!" "file"
\end{lstlisting}
\end{block}
\end{frame}


\begin{frame}[fragile]
\frametitle{Functional v Imperative}
\begin{block}{We can do this equational reasoning on imperative programs}
\textbf{because we are functional programming}
\end{block}
\end{frame}


\begin{frame}[fragile]
\frametitle{Functional v Imperative}
\begin{itemize}
  \item<1-> Functional Programming
  \item<2-> or Imperative Programming
  \item<3-> \textbf{but never both}
\end{itemize}
\visible<4->{
  \begin{tikzpicture}[remember picture,overlay]
  \coordinate (aa) at ($(a1)+(0,4.7)$);
  \node[right] at (aa) {\includegraphics[height=3cm]{image/bullshizzles.png}};
  \end{tikzpicture}
}
\end{frame}


\begin{frame}[fragile]
\frametitle{Functional v Imperative}
\begin{block}{Functional programming is}
just a \tiny{not ridiculous}\normalsize{ means of imperative programming}
\end{block}
\end{frame}


\begin{frame}[fragile]
\frametitle{Functional v Imperative}
\begin{block}{To further help demonstrate this point}
\begin{itemize}
  \item<1> pure-functional random value library using the C\# programming language
  \item<2> pure-functional RDBMS library using the Java programming language
  \item<3> pure-functional terminal I/O programs using the Ruby programming language
\end{itemize}
\end{block}
\end{frame}


\begin{frame}[fragile]
\frametitle{Functional v Imperative}
``I don't think the benefits of enforcing side-effect-free code (even optionally) make up for the many work-arounds you have to use to get anything done in the real world.''
\begin{center}
\includegraphics[width=0.1\paperwidth]{image/lalala.png}
\end{center}
``[Functional Programming makes] the code pretty unreadable for people who aren't used to functional''
\end{frame}

% 
