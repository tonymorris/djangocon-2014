{
  \newmdenv[tikzsetting={draw=black,fill=white,fill opacity=0.7, line width=4pt},backgroundcolor=none,leftmargin=0,rightmargin=0,innertopmargin=4pt,skipbelow=\baselineskip,%
  skipabove=\baselineskip]{TitleBoxWhatDoesFunctionalProgrammingMean}

  \usebackgroundtemplate{\includegraphics[width=1.0\paperwidth]{image/title-background.png}}

  \begin{frame}[plain] 
  \title{What is Functional Programming?}
  
  \vspace{3em}

  \begin{TitleBoxWhatDoesFunctionalProgrammingMean}
    \begin{center}
    {\Large \inserttitle}
    \end{center}
  \end{TitleBoxWhatDoesFunctionalProgrammingMean}

  \end{frame}
}


\begin{frame}
\frametitle{Functional Programming}
\begin{block}{Split the question in two}
\begin{itemize}
\item what does functional programming mean in principle?
\item what are the consequences of this principle?
\end{itemize}
\end{block}
\end{frame}


\begin{frame}
\frametitle{Functional Programming}
\begin{block}{What does functional programming mean?}
\begin{itemize}
\item<1> programming with functions
\item<2> yeah right, but what is a function?
\end{itemize}
\end{block}
\end{frame}


\begin{frame}
\frametitle{Functional Programming}
\begin{block}{A function}
relates every argument to a result \textbf{and does nothing else}
\end{block}
\end{frame}


\begin{frame}
\frametitle{Functional Programming}
\begin{block}{Functions give rise to \emph{referential transparency}}
An expression \lstinline$expr$ is referentially transparent if in all programs \lstinline$p$,
all occurrences of \lstinline$expr$ in \lstinline$p$ can be replaced by the result assigned
to \lstinline$expr$ without causing an observable effect on \lstinline$p$.
\end{block}
\end{frame}


\begin{frame}[fragile]
\frametitle{Functional Programming}
\begin{block}{Functions give rise to \emph{referential transparency}}
\begin{lstlisting}[style=python,mathescape]
def p(): 
  x = `expression`
  proc(x, x)
\end{lstlisting}
\begin{tikzpicture}[remember picture,overlay]
\coordinate (aa) at ($(a1)+(5,5.2)$);
\node[note,draw,callout relative pointer={($(aa)-(7.8,2.2)$)},right] at (aa) {is \lstinline$expression$ referentially transparent?};
\end{tikzpicture}
\end{block}
\end{frame}


\begin{frame}[fragile]
\frametitle{Functional Programming}
\begin{block}{Functions give rise to \emph{referential transparency}}
\begin{lstlisting}[style=python,mathescape]
def p(): 
  # x = expression
  proc(`expression`, `expression`)
\end{lstlisting}
\begin{tikzpicture}[remember picture,overlay]
\coordinate (aa) at ($(a1)+(2,2.0)$);
\node[note,draw,callout relative pointer={($(aa)-(4.4,-3.1)$)},right] at (aa) {has this refactoring affected the program?};
\node[note,draw,callout relative pointer={($(aa)-(1.8,-3.1)$)},right] at (aa) {has this refactoring affected the program?};
\end{tikzpicture}
\end{block}
\end{frame}


\begin{frame}[fragile]
\frametitle{Functional Programming}
\begin{block}{Referential Transparency}
\begin{lstlisting}[style=python,mathescape]
def print2(s, t):
  print(s)
  print(t)

def strpopthen():
  s = 'abcdef'
  x = `s[0]`
  print2(x,x)
\end{lstlisting}
\begin{tikzpicture}[remember picture,overlay]
\coordinate (aa) at ($(a1)+(5,4.2)$);
\node[note,draw,callout relative pointer={($(aa)-(8.1,2.5)$)},right] at (aa) {referentially transparent?};
\end{tikzpicture}
\end{block}
\end{frame}


\begin{frame}[fragile]
\frametitle{Functional Programming}
\begin{block}{Referential Transparency}
\begin{lstlisting}[style=python,mathescape]
def print2(s, t):
  print(s)
  print(t)

def strpopthen():
  s = 'abcdef'
  # x = s[0]
  print2(`s[0]`,`s[0]`)
\end{lstlisting}
\begin{tikzpicture}[remember picture,overlay]
\coordinate (aa) at ($(a1)+(1,1.5)$);
\node[note,draw,callout relative pointer={($(aa)-(1.3,-1.9)$)},right] at (aa) {program changed?};
\node[note,draw,callout relative pointer={($(aa)-(0.3,-1.9)$)},right] at (aa) {program changed?};
\end{tikzpicture}
\end{block}
\end{frame}


\begin{frame}[fragile]
\frametitle{Functional Programming}
\begin{block}{Referential Transparency}
\begin{lstlisting}[style=python,mathescape]
def print2(s, t):
  print(s)
  print(t)

def listpopthen():
  s = ['a','b','c','d','e','f']
  x = `s.pop()`
  print2(x,x)
\end{lstlisting}
\begin{tikzpicture}[remember picture,overlay]
\coordinate (aa) at ($(a1)+(5,2.8)$);
\node[note,draw,callout relative pointer={($(aa)-(7.7,0.1)$)},right] at (aa) {referentially transparent?};
\end{tikzpicture}
\end{block}
\end{frame}


\begin{frame}[fragile]
\frametitle{Functional Programming}
\begin{block}{Referential Transparency}
\begin{lstlisting}[style=python,mathescape]
def print2(s, t):
  print(s)
  print(t)

def listpopthen():
  s = ['a','b','c','d','e','f']
  # x = s.pop()
  print2(`s.pop()`,`s.pop()`)
\end{lstlisting}
\begin{tikzpicture}[remember picture,overlay]
\coordinate (aa) at ($(a1)+(1,1.5)$);
\node[note,draw,callout relative pointer={($(aa)-(1.3,-1.8)$)},right] at (aa) {program changed?};
\node[note,draw,callout relative pointer={($(aa)-(-0.3,-1.8)$)},right] at (aa) {program changed?};
\end{tikzpicture}
\end{block}
\end{frame}


\begin{frame}
\frametitle{Functional Programming}
\framesubtitle{What does functional programming mean?}
\begin{block}{The essence of functional programming}
is the demand that expressions, \textbf{in general}, maintain referential transparency
\end{block}
\end{frame}


\begin{frame}
\frametitle{Functional Programming}
\framesubtitle{but why?}
\begin{block}{referential transparency gives rise to, \textbf{and monopolises}}
\begin{itemize}
\item \emph{equational reasoning}

      an essential tool for code readability
\item \emph{modularity}

      delineating concepts, building new programs from slightly smaller programs
\end{itemize}
\end{block}
\end{frame}


\begin{frame}
\frametitle{Functional Programming}
\framesubtitle{therefore}
\begin{block}{Functional Programming is}
\begin{itemize}
\item<1> a thesis, independent of any programming language
\item<2> to some extent, a programming language may provide the programmer assistance in achieving adherence to the thesis
\item<3> not anything more than this, despite what you may have been told
\end{itemize}
\end{block}
\end{frame}


\begin{frame}
\frametitle{Functional Programming}
\framesubtitle{python}
\begin{block}{Well that raises an interesting question}
Does python assist in achieving this objective?
\end{block}
\end{frame}


\begin{frame}
\frametitle{Functional Programming}
\framesubtitle{python}
\begin{center}
\huge{No.}
\end{center}
\normalsize
\end{frame}


\begin{frame}
\frametitle{Functional Programming}
\begin{block}{This question has been explored and answered}
If anyone tells you that python supports functional programming to any (non-trivial) extent, \textbf{they are outright lying to you}
\end{block}
\tiny{and I have the code to prove it}
\end{frame}


\begin{frame}
\frametitle{Functional Programming}
\begin{block}{As a consequence}
Python demands that you, the programmer, forgo many of the most essential tools of progressive software development
\end{block}
\end{frame}
